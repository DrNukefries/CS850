\documentclass[11pt]{article}

\usepackage{color}
\usepackage{url}
\usepackage[margin=0.7in]{geometry}
\usepackage{enumitem}
\usepackage[colorlinks=true, bookmarks=true,  plainpages=false, pdfpagelabels=true, pageanchor=false,hypertexnames=false,linkcolor=blue,citecolor=blue,pdfauthor={Marek Petrik},urlcolor=blue]{hyperref}
\usepackage{booktabs}
%\usepackage{gfsdidot}
\usepackage[T1]{fontenc}
%\usepackage{newpxtext,mathpazo}

\setlength{\parskip}{3mm plus1mm minus1mm}
\setlength{\parindent}{0cm}

\date{}

\begin{document}
\begin{center}
	{\Large\bf CS 750/850: Machine Learning}\\
	Fall 2025\\
	\textbf{Website:} \url{https://gitlab.cs.unh.edu/carton/ml-fall2025}
\end{center}


\section{Basic Course Information}

\begin{itemize}
\item \textbf{Department}: Computer science
\item \textbf{Course title}: CS 750 and CS 850
\item \textbf{Number of credits}: 4 for CS 750 and 3 for CS 850
\item \textbf{Semester}: Fall 2025
\item \textbf{Modality}: In person. I will broadcast the stream over zoom by request for any student
who is sick or otherwise unable to be on campus, but I don't do it by default.
\end{itemize}

\section{Basic Instructor Information}

\begin{itemize}
\item \textbf{Name of the instruction}: \emph{Samuel Carton}
\item \textbf{Email address}: \url{mailto:samuel.carton@unh.edu}
\item \textbf{Office hours}: Available on Canvas
\item \textbf{Names and contact of TA}: Available on Canvas
\item \end{itemize}

The \emph{fastest} way to get your question about course material answered is to post it on the discussion forum listed on the website.
Please do not email the instructor or the TAs with questions that may also be of interest to others.
If you have personal questions, it is best to come to the office hours.

\section{Description of the Course}

\subsection{Introduction}
This class will teach you how to use machine learning to understand data and make predictions. It focuses on understanding the fundamental concepts and algorithms that underlie modern machine learning and data science.

\subsection{Student learning outcomes}
Learn basic machine learning techniques, learn how they are derived from statistical principles, and learn how they are solved using linear algebra and mathematical optimization.

\subsection{Format of the course}
Lectures, assignments, and online discussion.

Class attendance is important for your learning. You are responsible for all course
assignments and meeting all deadlines unless exceptions are agreed upon with the
instructor ahead of time. Attendance in this course is optional.

\section{Prerequisites}

Students need to be able to code in a scripting language, such as Python, R, or Julia.
Ideally, the students will have also completed a course on statistics.
Some experience with linear algebra will also be helpful.

\section{Course Requirements}

\subsection{Nature of assignments and exams}

There will be homework assignments due approximately every week.
The assignments can be turned in either in the class or online.
The assignments will involve both coding and theoretical components.

The exams (mid-term and final) will be in person.
The midterms will take place halfway through the semester and the specific date will be announced at least two weeks in advance.
The final exam will be scheduled according to UNH policies.

The class also involves a class project which includes several deliverables and culminates in a presentation either in
the form of a poster or an oral presentation. Most of the project work will happen in October.

\subsection{Federal credit hour requirement}

The class is expected to take about 12 hours of work a week including lectures, reading the study materials, and completing assignments.

\subsection{Deadlines and test dates}

There will be an in-person midterm exam and an in-person final exam. These will be \textbf{closed-book} exams.
Please see the class website and MyCourses for specific dates for the exams, assignments, and quizzes.
No plagiarism is allowed when completing exams.

\subsection{Description of grading procedures}

\paragraph{Late policy}
\begin{enumerate}[nosep]
\item \textbf{Late penalty:} 5\% per \textbf{hour}.
\item Start early. Do not put it off until the last moment.
\end{enumerate}

\paragraph{Assignments}
\begin{enumerate}[nosep]
    \item One worst grade in assignments 1-7 and another one in assignments 8-14 will be dropped.
    \item If you have a legitimate reason for why you cannot finish an assignment on time, please talk to the instructor.
    \item Assignments are due on the day and hour indicated on myCourses. Generally due on Tuesdays.
    \item Assignments should be turned in as a \textbf{PDF} and a \textbf{separate source code file} (Quarto, unless that does not work) on myCourses. There may be no credit for submitting only source code that does not compile.
    \item Please do not copy solutions from online sources or your friends; it defeats the purpose of the assignments and you will not learn the material.
    \item We encourage you to collaborate with your classmates, but this must not involve copy-pasting. Copy-pasting from online sources is also not OK. Each homework solution should be prepared independently.
\end{enumerate}

The class will involve hands-on data analysis using machine learning methods. The recommended language for programming assignments is R which is an excellent tool for statistical analysis and machine learning. No prior knowledge of R is needed or expected; the book and lectures will include a gentle introduction to the language. Python is also an option for students who have some experience with it and are willing to figure things out on their own.

\paragraph{Quizzes} The quizzes will be available on MyCourses. There will be 3 attempts for each quiz, with the highest grade counting. 


\paragraph{Project} The class will involve a project with multiple groups of up to 5 people working as a team. The class project will have a longer report and presentation due at the end of the semester.

\paragraph{Final grade}
The final grades will be computed using the standard UNH scale.

\subsection{Components of final grade}

The weights of individual components in computing the grade are as follows:
\begin{itemize}
\item \textbf{20\%} Homework assignments  
\item \textbf{10\%} Quizzes 
\item \textbf{10\%} Class project 
\item \textbf{20\%} Midterm exam  
\item \textbf{40\%} Final exam 
\end{itemize}

\section{Learning Resources}

\subsection{Textbooks}

\begin{enumerate}

\item James, G., Witten, D., Hastie, T., \& Tibshirani, R. (2021). \emph{An Introduction to Statistical Learning}. \textbf{second edition}. The pdf is available online.

\item Deisenroth, M. P., Faisal, A. A., \& Ong, C. S. (2021). \emph{Mathematics for machine learning}. The pdf is available online.
\end{enumerate}

\subsection{Campus resources}

\section{Course Policies}

\subsection{University-based policies}

\paragraph{Disability}

If you are registered with the student disability office, please let me know right away so that I can provide proper accommodation.

According to the Americans with Disabilities Act (as amended, 2008), each student
with a disability has the right to request services from UNH to accommodate their
disability. If you are a student with a documented disability or believe you may
have a disability that requires accommodations, please contact Student Accessibility Services (SAS) at 227 Smith Hall, or at sas.office@unh.edu.

Accommodation letters are created by SAS with the student. Please follow up with
your instructor as soon as possible to ensure the timely implementation of the identified accommodations in the letter. Faculty need to respond once they receive official notice of accommodations from SAS, but are under no obligation to provide retroactive accommodations.

For more information, contact SAS: 227 Smith Hall, www.unh.edu/sas,
603.862.2607, 711 (Relay NH) or sas.office@unh.ed

\paragraph{Academic honesty}

Please, be honest. No forms of cheating or plagiarism will be tolerated. You are encouraged to talk to your classmates about the problems as well as search the Internet. \textbf{However, please do not seek out solution guides, and if you come across one do not study it.} It is \textbf{not} OK to copy code from classmates, the internet, or anywhere. If you have the same variable names, exactly the same errors, same comments as another classmate or an online source, your solution will be considered copied and you \underline{will fail the class}.  \textbf{In 2020, 5 students failed the class because of plagiarism and 3 of them had to leave UNH.}

\begin{itemize}[noitemsep]
\item University Academic Honesty Policy:\\
     \url{https://www.unh.edu/student-life/handbook/academic/academic-honesty}
\item Tutorial on Plagiarism: \\
    \url{http://cola.unh.edu/plagiarism-tutorial-0}
\end{itemize}


\subsection{Course specific policies}

\paragraph{Use of Large Language Models (such as ChatGPT)} You can use LLMs to help with the completion of the assignments.
The final work needs to be your own and you need to declare the sources used when preparing the assignments
(including ChatGPT and any online resources).
There are no online sources allowed during exams except those explicitly stated.
Since the exams will be closely related to the assignments, I recommend working out the assignments on your own as much as possible,
rather than submitting generated work.

\paragraph{Exceptions} If you have a serious reason for not being able to complete an assignment or a quiz, or
have other concerns about succeeding in the class, please do not hesitate to contact the instructor.

\section{Schedule}

\subsection{Course topics}

\begin{enumerate}
\item Basic Probability
\item Statistics
\item Linear regression 
\item Basic linear algebra and calculus
\item Linear algebra: Implementing linear regression
\item Logistic regression
\item Generative models
\item Generalized linear models
\item Generalized models and log-likelihood
\item Cross-validation
\item Bayesian linear regression]
\item Regression splines
\item Decision trees and boosting
\item SVM
\item Optimization
\item Dimensionality reduction: PCA
\item k-means
\item EM
\item Neural nets
\item Fitting neural nets and optimization
\item Convolutional and recurrent neural nets
\item Transformers and recommender systems
\end{enumerate}

\subsection{Due dates}

Please see the class website for the most current due dates.

\subsection{Readings}

Please see the class website for the recommended reading for each lecture.

% -------------------------------------------

\end{document}

%%% Local Variables:
%%% mode: latex
%%% TeX-master: t
%%% End:
